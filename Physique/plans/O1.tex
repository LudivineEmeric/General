\Chapter{Laser}

\nivpre{L3}{
 \begin{itemize}
 \item Oscillateurs électroniques
 \item Equation de Schrödinger
 \item Electromagnétisme dans les milieux diélectriques
 \item Interféromètres de Michelson et Fabry Pérot
 \item Effet Doppler
 \item Forme des orbitales atomiques
 \end{itemize}
}

\lemessage{..}

\biblio{}

\section*{Introduction}

LASER : Light Amplification by Stimulated Emission of Radiation \\
Source lumineuse omniprésente et très utile : Cohérente, monochromatique, faisceau parallèle et concentré

\marginpar{\cite{ILM_Faroux}}
\marginpar{\cite{Optique_houard} ch11}
\marginpar{\cite{Optique_sextant} ch4}
\marginpar{\cite{laser_dangoisse}}
\marginpar{\cite{ToutenunPC_2016} ch30-31}


\section{Principe de fonctionnement}
\subsection{Rappel oscillateur électronique}
Condition de Barkhausen
\subsection{Principe du Laser}

\section{Amplification par émission stimulée}
\subsection{Système à deux niveaux}
Hyp : 2 niveaux d’énergies non dégénérés (a, b) interagissent avec le champ électromagnétique
Fonction d’onde du système : ...

\subsection{Coefficients d’Einstein. Emission stimulée}
Grâce au développement de la partie précédente nous pouvons obtenir l’équation d’Einstein \\
Condition pour avoir émission stimulée ...
\subsection{Pompage}
\section{Cavité résonnante}
\subsection{Cavité Fabry Pérot}
\subsection{Application : le LIDAR}
exploite l’effet doppler
\subsection{Le Laser stationnaire}
Point de fonctionnement du laser et raies transmissent par le laser...

\section*{Conclusion}
Laser : oscillateur optique \\
Amplification par émission stimulée découle de l’eq de Schrödinger \\
Filtrage peut être effectué par une cavité Fabry-Pérot \\
Ouverture sur les lasers semi-conducteurs \\


\begin{remarques}
\begin{itemize}
\item nb de pics et largeur de raies typique d’un laser : dépend de la largeur du gain et de la largeur des pics issus du Fabry-Pérot $10^10 Hz$ à $10^15 Hz$. On peut mettre un filtre pour sélectionner le mode qui nous intéresse (filtre de lLyot) et rendre aussi le laser monomode. Pour rendre un laser monomode on peut aussi jouer sur la taille de la cavité
\item taille cavité FP laser semiconducteurs : ordre du mm au μm (rq pour laser He-Ne de l’ordre de 17cm). Pour ces lasers le gain est plus relativement homogène, il est plus facilement monomode
\item laser à solide : milieu = cristal (par ex Al2O3) les électrons sont excités, pertes non radiatives dans le réseau cristallins puis désexcitation. Le gain est plus homogène. Le gain sature sur l’ensemble de la courbe et un seul mode lase. Pour un laser à gaz, le gain est inhomogène et le gain va saturer pour chaque groupe d’atomes faisant partie de la même classe de vitesse. Plusieurs modes vont pouvoir laser. Le plus connu Nd:YAG (ophtalmologie, nettoyage de façade cf : c’est pas sorcier)
\item laser les plus connus et applications : He-Ne, Argon, Xénon, Crypton, Diode laser (lecteur CD), semi-conducteur (utilisés dans toutes les télécommunications, fibre optique, téléphone, internet…). Rq : les lasers à gaz ne sont pas les plus utilisés, juste pour l’enseignement et en métrologie car ils sont plus stable. Mesure distance Terre-Lune, LIDAR, Correction de la myopie
\item laser à impulsion : on apporte de l’énergie en permanence (le gain augmente, mais on s’arrange pour qu’il y
ait plus de pertes que de gain). En pratique on peut utiliser des cellules de pockels dont on peut contrôler l’indice optique et la biréfringence avec la tension appliquée. en les associant avec un polariseur à $45^o$ des axes neutres, on peut en faire des interrupteurs contrôlés électriquement. On arrive ainsi à obtenir une très forte inversion de population et d’emmagasiner de l’énergie dans le milieu à gain. On commute les pertes $\rightarrow$ Tres forte inversion de population pour ramener le gain = pertes, libère bcp de photon
\item revoir gain en fonctionnement stationnaire, gain quand I non nul, gain en fonction de I
\end{itemize}
\end{remarques}

%
%\begin{ecran}
%	Contenu affiché sur diapositives
%\end{ecran}
%

%
%\transition{Belle transition entre deux parties ou sous-parties.}
%
%
%
%\begin{experience}
%	Expérience pour illustrer
%\end{experience}
%
%\begin{attention}
%	Erreur faite souvent ou point sur lequel insister.
%\end{attention}