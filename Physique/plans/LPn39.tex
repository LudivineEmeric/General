\Chapter{Production et analyse de la lumière polarisée}

\nivpre{L3}{
 \begin{itemize}
  \item aspect ondulatoire de la lumière
  \item physique des ondes
  \item polarisation rectiligne et loi de Malus
 \end{itemize}
}

\lemessage{Lame quart d'onde et lame demi-onde au programme de PC.}

\biblio{}

\section*{Introduction}
limites de l'optique géométrique \\
interférences et diffraction : on a mis de côté l'aspect vectoriel de la lumière \\
on s'y intéresse ici : description, production et analyse \\

\marginpar{réf. \cite{ToutenunPC_2016},\cite{hecht},\cite{huard_pola}}

\section{Production de lumière polarisée}
\subsection{Etats de polarisation de la lumière}
Onde plane progressive harmonique OPPH \\
Expression du champ électrique \\
déphasage x,y dépend de t en générale : émission thermique, lampe à incandescence est aléatoire par exemple ; pour laser : déphasage constant \\
on a alors : polarisation rectiligne, circulaire et elliptique \\
vidéo Youtube avec propagation \\

\subsection{Production de lumière polarisée par absorption}
absorption : polariseur \\
dichroïque \\
visible : absorption par charges accélérées dans le matériau (en fait il y a peu de transmission, bcp de réflexion) \\


\subsection{Production par diffusion}
rayonnement dipolaire \\

\subsection{Production par réflexion vitreuse}
réflexion sur un dioptre \\
coefficients de Fresnel \\
angles de Brewster en polarisation TM : $\tan i_B = \frac{n_1}{n_2}$ 
exemple pour le verre : $i_b=56^o$ \\



\section{Analyse d'une lumière polarisée}


\subsection{Biréfringence}
Milieu isotrope : toutes directions ont même indice \\
Milieu biréfringent : définition, milieu qui possède deux indices de réfraction \\
soit uniaxe (no,no,ne), soit biaxe (no,ne,ne -- 2 axes optiques) \\
ex : le quartz est un milieu uniaxe \\
ie indice ordinaire et indice extraordinaire \\
ce qui nous intéresse ici : lame taillée parallèlement à axe extraordinaire \\
axes neutres, bien définir les axes \\
calcul $Delta\varphi$ \\
Lame quart d'onde : $\delta=\lambda/4$

\subsection{Action d'une lame quart d'onde}
incident rectiligne \\
cas général : polarisation elliptique \\
si $\alpha=0(\pi/2)$ pola rectiligne, si $\alpha=\pi/4(\pi/2)$ circulaire \\
incident circulaire : après lame pola rectiligne \\
incident elliptique : pola rectiligne

\section*{Conclusion}
mind map disjonction des cas \\
application : lunettes de soleil, ailleurs dans le spectre électromagnétique : antennes radio \\


\begin{remarques} \begin{itemize} 
\item la polarisation intervient dans interférences
\item onde plane : la surface d'onde est plane 
\item progressive : elle se propage dans une direction
\item on peut avoir des ondes stationnaires : pas de propagation apparente (compensation dues à conditions aux limites)
\item harmonique : une longueur d'onde = monochromatique
\item source idéale, laser s'en rapproche le plus
\item une source dans la pièce, peut on considérer onde plane ? non, quantifier avec atténuation, l'amplitude décroit en A/r
\item onde plane pour source éloignée : il faut $r>lambda$ et $r>a$
\item k E B trièdre : toujours vrai ? non : c'est Poynting, E, H ; k, D, B ; k et E ne sont pas orthogonaux dans un milieux avec des charges
\item elliptique droite ou gauche ? si on prend dans le plan $z=0$, on regarde le déplacement avec le temps, composantes $E_{0,x}$ et $E_{0,y} \cos (\varphi)$ ; $\varphi$ entre 0 et $\pi$ : gauche ; entre 0 et $-\pi$ : droite
\item polarisation par diffusion : dipôle oscille, pas d'émission dans l'axe du dipôle, il n'y a pas de composante orthoradiale 
\item réflexion vitreuse dans le cadre de programme PC, diffusion aussi
\item axe neutre d'une lame : une polarisation rectiligne reste rectiligne
\item lame à retard : définie à une longueur d'onde donnée
\item $\Delta\varphi=\frac{2\pi}{\lambda} \Delta n \, e$ : plus $\Delta n \, e$ est grand, moins on est sensible à $\lambda$, tant que multiple impair de $\frac{\lambda}{4}$
\item voir animation Arnaud + code Python
\item comment dépolariser la lumière : sur un spectre large une $\lambda/4$ avec une grande épaisseur ou sur un laser activité optique différente spatialement (prisme de Wollaston, Babinet)
\item $\frac{\lambda}{2}$ est intéressante aussi : rotation d'une polarisation rectiligne 
\item Brewster astucieusement : un dipôle dans la matière est excité, il n'y a pas de réémission dans l'axe
\item uniaxe positif : $n_e>n_0$, axe lent : axe extraordinaire (vitesse plus faible) = AO (vitesse radiale e<o)
\item Negatif : ne<no
\item Les axes optiques sont les intersections des nappes ordinaire et extraordinaires (apparaît comme milieu isotrope sur ces axes)
\item Biaxe : 2 AO ; Uniaxe : 1AO
\item prisme de Wollaston
\end{itemize} \end{remarques}