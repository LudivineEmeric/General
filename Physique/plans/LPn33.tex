\Chapter{Collisions et lois de conservation en mécanique classique et relativiste}

\nivpre{L3}{
 \begin{itemize}
  \item mécanique du point
  \item changement de référentiel
  \item relativité, quadrivecteurs, loi de la dynamique
 \end{itemize}
}

\lemessage{Socle de l'étude de la dynamique des réactions nucléaires et des collisions de particules (désintégration, collision, etc.).}

\biblio{}

\section*{Introduction}
il y a partout des collisions \\
on reste sur le cas élastique (conservation énergie cinétique, pas transfert énergie interne, cf boules de billard) inélastique=déformation en général (cf 2 voitures) \\
montrer pendule de Newton \\
plein d'images \\
\href{https://www.youtube.com/watch?v=4v2RHtBTbj8}{Vidéo Youtube du choc de 2 billes} \\
faire que des points matériels ou sphères dures \\
A quoi servent les collisions ? sonder la matière \\

\section{Conservation de l'énergie}
exp de rutherford : particule alpha sur cible particule Au, totalement classique (une particule ponctuelle, l'autre non et on détermine rayon de l'atome d'Au) \\
on sonde la matière qu'on fait collisionner \\

\section{Conservation du quadrivecteur énergie-impulsion}
diffusion compton : photon sur électron, diffusion \\


\bigskip
Seulement si le temps : 
\section{Collision inélastique : boson de Higgs}
découverte du boson de Higgs \\
2 gluons fusionnent, se désintègre en 2 photon (symétriques), 2 hbar k/c, à partir du pic d'énergie en fct de masse \\
trouver figure sur google

\section*{Conclusion}


% conservation de l'énergie
% 		exp de rutherford : alpha sur particule en Au, totalement classique (une particule ponctuelle, l'autre non et on détermine rayon de l'atome d'Au), on sonde la matière qu'on fait collisionner
%					modèle sphères dures ? solides non déformables ok
% conservation du quadrivecteur énergie-impulsion 
% 		diffusion compton : photon sur électron, diffusion d'un photon
% collision inélastique : découvrte du boson de Higgs
			% 2 gluons fusionnent, se désintègre en 2 photon (symétriques), 2hbark/c, à partir du pic d'énergie en fct de masse

\begin{remarques} \begin{itemize} 
\item en pratique, pour analyser une réaction, on fait un usage intense du fait que, d’une part, le quadrivecteur énergie-impulsion est conservé, et que, d’autre part, les pseudo-produits scalaires de quadrivecteurs sont invariants par changement de référentiels
\item référentiel du centre de masse : impulsion totale est nulle (mais pas l'énergie)
\item la masse invariante M est un invariant de Lorentz : pour un système donné, sa valeur est indépendante du référentiel.
\item certaines collisions inélastiques en classique sont élastiques en relativiste
\item symétrie $\Rightarrow$ conservation d'une quantité
\item ne pas faire trop de calculs
\item ne pas dissocier classique et relativiste, traiter ensemble
\end{itemize} \end{remarques}