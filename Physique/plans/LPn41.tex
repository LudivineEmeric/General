\Chapter{Phénomènes de diffusion}

\nivpre{CPGE}{
 \begin{itemize}
  \item premier principe de la thermodynamique
  \item bilans locaux (d'énergie, ttc...)
 \end{itemize}
}

\lemessage{.}

\biblio{}

\section*{Introduction}
terme du langage courant, qu'est-ce que ça veut dire en physique ? \\
Vidéo diffusion : diffusion microscopique \\
application : isolation habitat, résistance thermique \\


\marginpar{réf. \cite{ToutenunPC_2016},\cite{Diu_thermo}}

\section{Diffusion de particules}
\subsection{}
Bilan \\
Diffusivité en $m^{2}.s^{-1}$
\subsection{Loi de Fick}
phénoménologique \\
irréversibilité \\
ex : diffusion parfum : $D\approx10^{-6}-10^{-4}$, $\tau=l^2/D=115\,jours$ à une distance de 10m \\
sucre dans eau : $D\approx10^{-12}-10^{-8}$, café $\tau=10^{11}s$ \\
phénomène très lent
\subsection{Équation de diffusion}


\subsection{Analogie thermique}
On a la même chose !! Tableau \\
phénomènes lents à nos échelles \\

\subsection{Diffusion thermique}
(pression constante car s'équilibre avec l'environnement, système est difficile à définir, éviter de dire des bêtises, utiliser H (voir \cite{Diu_thermo})ou juste analogie) \\
(conditions aux limites pas évidentes) \\

\subsubsection{Densité de courant thermique et équation de conservation}
Flux thermique \\
Établissement de l'équation de conservation 1D \\
\subsubsection{Loi de Fourier et équation de diffusion}
même chose avec diffusivité thermique \\
OdG conductivité thermique latériaux non métalliques $\sim 1\,W.s^{-1}.K^{-1}$, métaux $\sim 100\,W.s^{-1}.K^{-1}$
\subsubsection{En régime permanent : détermination de la conductivité thermique}
expérience avec cuivre \\
résultat : $460\,W.s^{-1}.K^{-1}$

\section{Aspects microscopiques}
Marche aléatoire (voir cours Alain Aspect) \\
relation d'Einstein \\

\section{Applications}
voir Termniale STI2D \\
résistance thermique \\
résistance hydraulique \\

\section*{Conclusion}
Tableau récapitulatif des deux diffusions \\
Généralisation \\
Plus tard : en mécanique des fluides diffusion de la quantité de mouvement \\


\begin{remarques} \begin{itemize} 
\item résistance thermique $\Delta T = R_{th} \Phi$, analogue à loi d'Ohm
\item dépend de la surface ou du contour de la surface ?
\item définir problème 1D : $j_Q$ ne dépend que d'une variable spatiale ; plusieurs dimensions : surface peut changer, étudier le rapport ? 
\item barre unidimensionnelle : il y a des pertes latérales, donc ne dépend pas d'une dimension spatiale
\item intéressant : calcul marche aléatoire pour établir diffusion
\item différence conductivité thermique métaux/non-métaux car conductivité électrique
\item renversement du temps : solution n'est plus la même, donc pas réversible
\item exemple de source de diffusion : réaction chimique, création d'une molécule ; réaction nucléaire (source de particules et de température) ; photon dans soleil
\item exemple où Fick ne marche pas : supernova, bombe nucléaire 
\item 
\item diffusion de matière, diffusion thermique, diffusion de quantité de mouvement, diffusion de charge
\item relation d'Einstein : fluctuation-dissipation
\item milieux poreux chez BCPST
\item exemples en biophysique : livres PACES (on a des livres d'exos pour médecine \cite{biophysique}), diffusion dans gel, quantité de nutriments dans cellule (taille limite d'une cellule)  
\item 
\item 
\item HS : diffusion des ondes EM (élastique vs inélastique) : onde-particule (Thomson, Compton), onde-matière (Rayleigh, Mie, Raman)
\end{itemize} \end{remarques}