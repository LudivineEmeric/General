\Chapter{Manifestations du caractère non galiléen d'un référentiel}

\nivpre{Licence}{
 \begin{itemize}
  \item mécanique du point (cinématique)
  \item force d'attraction newtonienne
  \item changement de référentiel, point coïncident (=point fictif que l'on balade entre les référentiels)
  \item principe d'inertie
 \end{itemize}
}

\lemessage{Attention le référentiel terrestre n'est plus vraiment étudié en CPGE et changement de référentiel est en spé. Ne pas faire trop de théorie, juste rappel sur slide.}

\biblio{}

\section*{Introduction}

\section{Description des référentiels usuels}
\subsection{Principe d'inertie}
\subsection{Changement de référentiel et forces d'inertie}
Rappels
\subsection{Une définition relative}

\section{Application au référentiel terrestre}
\subsection{Force d'inertie d'entraînement}
\subsection{Force d'inertie de Coriolis}
\section*{Conclusion}



\begin{remarques} \begin{itemize} 
\item trop de rappels, balancer plutôt les lois de composition et faire un schéma pour le point H dans $-\Omega^2\vec{HM}$
\item autres exemples manifestations terrestre non galiléen : face de la lune est toujours la même, dérive des iceberg, Pluton et un de ses satellites sont toujours la même face l'une de l'autre, galaxies ont forme de spirale, Terre est élargie aux équateurs (implique $\Delta g=0,06$), déviation vers l'est, marées, cyclones
\item autres non galiléens : principe de l'essoreuse à salade, nucléaire pour séparer Rd... de Rd... ?
\item référentiel inertiel = référentiel galiléen
\item principe d'équivalence d'Einstein : masse inertielle et masse gravitationnelle sont les mêmes 
\item principe de relativité affirme que les lois physiques s'expriment de manière identique dans tous les référentiels inertiels : les lois sont « invariantes par changement de référentiel inertiel »
\item intéressant de parler des lavabos car idée reçue, vs cyclones, comparer dimensions... 
\item 
\end{itemize} \end{remarques}