\Chapter{Fermions, bosons, illustrations}

\nivpre{Licence}{
 \begin{itemize}
  \item distribution canonique et grand canonique
   \item traitement microcanonique du gaz parfait
   \item postulat de symétrisation mécanique quantique
   \item notion de spin
 \end{itemize}
}

\lemessage{.}

\biblio{}

\section*{Introduction}
aspects quantique du modèle microcanonique du gaz parfait \\

%\marginpar{réf. \cite{ToutenunPC_2016},\cite{hecht},\cite{huard_pola}}

\section{Formalisme des statistiques quantiques}
\subsection{Rappel : résultat semi-classique}
entropie de Sackur-Tetrode \\
$S>0$ : limite diluée, $n\Lambda^3 <= 1$ \\
quantique : recouvrement fonctions d'onde \\

\subsection{Formalisme des statistiques quantiques}
Pour un système de N particules identiques, les seuls états physiques sont antisymétriques car symétriques par échange de particules.  \\
Boson, fonction d'onde symétrique par échange de deux particules, exemples : photon, He4 \\
Fermion, fonction d'onde antisymétrique par échange de deux particules, exemples : électrons, proton \\
Le spin des fermions est demi-entier \\
Le spin des bosons est entier \\
exmple : 2 bosons dans 2 états a et b, écriture de l'état, + entre les deux fonctions d'ondes  \\
2 fermions dans 2 états a et b, écriture avec un - entre les deux fonctions d'ondes \\
Application : principe d'exclusion de Pauli, deux fermions ne peuvent pas être dna sle même état quantique \\

\subsection{Factorisation de la fonction de partition (ensemble grand-canonique)}
T et $\mu$ fixés, N particules \\
$\Xi=\Sigma e^{\beta \left( E_l - \mu N_\lambda \right) }=\Sigma \Pi ...$  \\

formule de $<N_\lambda>$ en fonction de la dérivée de $ln ( \xi_\lambda )$ \\


\section{Distribution de Fermi-Dirac et Bose-Einstein}
\subsection{Stat de Bose-Einstein}
boson : $N_\lambda=0, 1,..., \infty$ \\
calcul de $\xi_\lambda$ \\
\subsection{Stat de Fermi-Dirac}
fermion : $N_\lambda=0$ ou $1$ \\
calcul de $\xi_\lambda$ \\

tracé des deux caractéristiques en fonction de $\beta(E_\lambda - \mu)$ : tendent va la même valeur \\
tendent vers la statistique de Maxwell-Boltzmann, expression \\

\section{Applications}
\subsection{Rayonnement du corps noir}
une boîte cubique \\
conditions aux limites périodiques \\
quantification des vecteurs d'onde \\
calcul calotte sphérique \\
ne pas oublier les 2 états de polarisation \\
densité spectrale de modes \\
on trouve la loi de Planck \\
loi de Wien \\
tracé pour différentes températures \\

\subsection{Comportement des électrons libres d'un métal}
équation de Schrödinger \\
densité d'énergie, calcul, en $E^\frac{1}{2}$ \\
à $T=0K$, FD devient fonction de Heavyside \\
$N=A E_F ^\frac{3}{2}$, expression énergie de Fermi : de quelques eV à quelques 10eV \\

\section*{Conclusion}


\begin{remarques} \begin{itemize} 
\item bien définir les fonctions de partition, tous les termes
\item Sackur-Tetrode dans micro-canonique, en fait ici on a donné son expression dans l'ensemble canonique (fonction de N, V, T)
\item théorie des champs : théorème spin statistique (hors programme)
\item fonction d'onde : N fermions ou N bosons, déterminant de Slater
\item loi de Planck : équilibre thermodynamique
\item thermostat : parois
\item loi du déplacement de Wien 
\item condensat de Bose : tous les bosons sont dans le même état, dans l'espace des k, refroidissement avec champ magnétique ou laser, la température est déterminée en fittant la "queue de la distribution" de population
\item sphère de Fermi : sphère en vecteur d'onde qui correspond à l'énergie de Fermi
\item masse de Chandrasekhar : masse limite au-delà de laquelle il y a effondrement gravitationnel, devient étoile à neutron
\item supraconducteur : paire de Cooper
\end{itemize} \end{remarques}