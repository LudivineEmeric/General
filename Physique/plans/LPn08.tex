\Chapter{Équations d'onde}

\nivpre{CPGE}{
 \begin{itemize}
  \item électrocinétique : loi de Kirchoff
  \item électromagnétisme : eq de Maxwell, force de Lorentz
  \item équation de d'Alembert (établi dans le cas de la corde 1D par exemple)
  \item dérivée particulaire
 \end{itemize}
}

\lemessage{Ondes dans plein de domaines.}

\biblio{}

\section*{Introduction}
définition onde et équation d'onde \\
propagation sans transport moyen de matière dont les dépendances sont définies par une équation d'onde \\
équation d'onde : équation différentielle à dérivées partielles, établit une relation entre dépendance temporelle et dépendance spatiale  \\
\marginpar{\cite{ToutenunMP_2004}}

autres : \cite{olivier_ondes}, \cite{guyon_hydrodynamique}

\section{Milieu non dispersif}
\subsection{Câble coaxial}
établissement de l'équation de d'Alembert \\
prendre le temps d'expliquer le modèle, dimensions infinitésimales \\

\subsection{Solutions de l'équation}
def surface d'onde, les différentes solutions : planes, sphériques, progressives \\
\subsection{Équation de dispersion}
vitesse de phase, vitesse de groupe \\


\section{Milieu dispersif}
définition atténuation, absorption \\
dispersif=vitesse de phase dépend de omega, il n'y a pas forcément d'absorption \\
retour sur le cable coax avec résistances (équation des télégraphistes)  \\

\subsection{Équation de Klein-Gordon}
\marginpar{\cite{ToutenunMP_2004} p516}
\marginpar{\cite{ToutenunPC_2016} p1001}

modèle du plasma dilué (néglige interactions entre particules du plasma) non relativiste \\
gaz ionisé neutre, hypothèse vitesse ions négligeable car beaucoup plus lourd \\
on regarde en fait la moyenne des électrons : PFD sur un électron moyen \\
par exemple ionosphère \\

\subsection{Relation de dispersion}
vitesse de phase, vitesse de groupe \\

\section*{Conclusion}
aussi en quantique, équation de Schrödinger \\
équations non linéaire \\
solitons \\



\begin{remarques} \begin{itemize} 
\item une équation d'onde n'est pas forcément linéaire
\item onde évanescente : pas qu'en MQ, ne se propage pas, SPP le long de la pénétration, miroir à atome
\item Kichoff : nécessite ARQS (propagation des ondes dans le milieu plus rapide que variation de courant et tension, dépend de la taille du système)
\item attention Klein-Gordon c'est limite prépa
\item onde de surface : sujet riche
\item penser aux conditions aux limites pour le guidage : change modes de propagation
\item plan alternatif : I/Onde longitudinale dans plasma (1) Modèle plasma, 2) établir eq d'onde, 3) eq de Klein-Gordon), II/Onde de surface (le guyon p258 (combiner Euler, évolution isentropique, conservation...), TF -> eq de dispersion avec tanh, beaucoup de régimes limites, plein de choses à dire, \href{https://www.youtube.com/watch?v=95sQcSulRFM}{vidéo Youtube forme des ondes de surface}...), ccl: ondes solitaires (phénomène non-linéaire)
\item équation télégraphiste, corde avec rigidité (dérivée 4eme de la position apparaît), onde acoustique
\item Schrödinger n'est pas irréversible ! Vraie pour l'équation de la chaleur
\item autre équation d'onde : équation thermique (onde de chaleur)
\item phénomènes non-linéaires : chaîne d'oscillateurs, pendule pesant...
\end{itemize} \end{remarques}

	