\Chapter{Ondes évanescentes}

\nivpre{CPGE 2ème année}{
 \begin{itemize}
  \item électromagnétisme (tout le cours)
  \item mécanique quantique (équation de Schrödinger)
  \item optique géométrique
 \end{itemize}
}

\lemessage{Vu dans plusieurs domaines, ici on fait le lien entre tout, transversal.}

\biblio{}

\section*{Introduction}
qqch qui s'atténue très rapidement dans une zone spatiale \\
définition propre... (nécessiter de sonder pour observer l'existence d'une telle onde) \\
manip : laser He-Ne, dioptre hémisphérique (ENSP585) (propriété : l'incidence est toujours normale), réflexion interne totale à l'interface rectiligne : a priori il n'y a pas d'onde, mais si l'on prend un autre prisme hemisphérique que l'on place approximativement proche de l'autre, on a un rayon, donc il y a transmission \\
ça s'appelle réflexion interne frustrée \\ 
\href{https://fr.coursera.org/lecture/mecanique-quantique/8-2-la-reflexion-totale-interne-frustree-e5hyh}{Vidéo IOGS Manuel Joffre}

\section{Ondes évanescentes électromagnétiques}
\subsection{Définition et mise en évidence expérimentale}
\subsection{Explication théorique : dépasser les lois de Snell-Descartes}
expression angle réflexion totale interne théorique\\
Equation de Helmholtz (attention c'est bien dans l'espace de Fourier), diélectrique homogène uniforme\\
étude réfraction en EM : k parallèle à l'interface est identique, omega aussi, k perpendiculaire peut être complexe
\section{Ondes évanescentes au sens large}
\subsection{Élargir à d'autres domaines de la physique}
effet de peau, plasma
autres exemples en physique (pas sûr):  \\
ondes évanescentes thermiques \\
dissipation de la quantité de mouvement dans un fluide visqueux \\
\subsection{Ondes évanescentes de matière : effet tunnel}
marhce de potentiel \\
résolution dans les 3 zones \\
\section{Applications}
\subsection{Microscopie en champ proche}
SNOM \\
sonde : guide d'onde \\
\subsection{Microscope à effet tunnel}
STM \\
\subsection{Détection d'empreintes digitales}
réflexion interne frustrée \\
(également : couplage entre guides d'onde, détecteur de rosée...cf mon projet ETI IOGS)


\section*{Conclusion}
partout, application technologique intéressante

\begin{remarques} \begin{itemize} 
\item \href{https://hal.archives-ouvertes.fr/tel-02493813/document}{thèse de Claire Li} \cite{ClaireLi}, chap2
\item plasmon de surface et plasmon de volume : filtre, couleurs et compagnie..
\item imagerie par résonance de plasmon...
\item le test de grossesse : première application bioplasmonique (voir slides Jerome Wenger)
\item capteur à résonance plasmon de surface (configuraiton de Otto ou Kretschmann)
\item traitement photothermique du cancer
\item évanescent ne veut pas toujours dire décroissance exponentiel : exemple avec effet tunnel et barrière par rectangulaire (WKB...) ; autre exemple barrière coulombienne...
\item déplacement de Goos-Hiinchen (TD d'électromagnétisme de Jeremy Neveu)
\item effet de peau : dissipation, sinon pas de dissipation
\end{itemize} \end{remarques}