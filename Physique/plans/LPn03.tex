\Chapter{Application des semiconducteurs à l'électronique et/ou l'optique }

\nivpre{Licence}{
 \begin{itemize}
  \item physique statistique : distribution de Fermi-Dirac
   \item caractéristique d'une photodiode
 \end{itemize}
}

\lemessage{Énormément utilisé dans les capteurs : on module la conduction pour faire plein de capteurs différents.}

\biblio{}

\section*{Introduction}
plein d'applications, toute l'électronique qu'on utilise aujourd'hui \\

%\marginpar{réf. \cite{ToutenunPC_2016},\cite{hecht},\cite{huard_pola}}

\section{Propriétés des semiconducteurs}
\subsection{Bande de valence, bande de conduction et énergie de gap}
Structure de bande \\
pour un semiconduteur énergie de gap inférieure à $5eV$ \\
pour un isolant énergie de gap supérieure à $5eV$ \\
A $T=0$, tous les électrons sont dans la bande de valence \\
A $T>0$, certains électrons passent dans la bande de conduction \\
densité de porteurs : $n_e$ électrons présents dans le BC, $n_h$ trous dans la BV \\
en pratique $n_e=N_e e^{\left( \frac{E_g-E_c}{k_B T}\right)}$  et  $n_h=N_ h e^{\left( \frac{E_g-E_c}{k_B T}\right)}$ \\


\subsection{Dopage}
exemple : on part d'un silicium intrinsèque (4 électrons de valence) \\
on remplace des atomes de silicium par des atomes d'arsenic (5 électrons de valence, très peu)  \\

change énergie de Fermi \\
dopé p : niveau de Fermi plus proche du haut de la bande de valence \\
dopé n : niveau de Fermi plus proche du bas de la bande de conduction \\

2 types de dopage : n $\rightarrow$ on ajoute des électrons, p $\rightarrow$ on ajoute des trous \\


\section{La jonction PN}
\subsection{Présentation du phénomène}
accoller matériau dopé p et matériau dopé p \\
les électrons s'équilibrent sur la zone à la limite des 2 \\
vidéo Youtube : formation and properties of junction diode   \\
zone de déplétion : il y a ni trou, ni électron $\rightarrow$ la circulation est bloquée \\
appariation d'un champ dans la ZCE, ie différence de potentiel : $V_0 = E_F^N -E_F^P = \frac{k_B T}{e} \log{\frac{N_a N_d }{n_i^2}}$ \\
dans un circuit, on va donc polariser la jonction \\

\subsection{Polarisation de la jonction}
on applique une différence de potentiel à la jonction PN \\
on change donc la différence de potentiel : $V_0'=V_0-V_a$, supérieure ou inférieure à $V_0$ \\
 caractéristique d'une diode \\
 illustration expérimentale avec une photodiode en changeant la résistance \\

\section{Applications}
\subsection{Application à la photodétection de lumière : la photodiode}
$V_a<0$ : polarisation en inverse \\
comment ça marche ? \\
on récupère les électrons "produits" du côté zone p \\
caractéristique : un photon excite un électron : création paire électron/trou \\
définition courant photonique \\

\subsection{Application à l'émission de lumière : la DEL (ou LED)}
$V_a>0$ : polarisation directe \\
électrons injectés du côté zone n \\
on peut faire des lasers \\
vidéo fonctionnement \\

\subsection{Application à l'électronique : le redresseur}
signal sinusoïdal envoyé à une diode avec une résistance de protection \\
lien avec signe diode, redressement \\

\section*{Conclusion}
on a présenté le cas le plus simple : jonction PN \\
on peut aussi faire PNP et NPN : amplificateur de puissance  \\
on peut faire thermistance car passage des électrons est régit par une distribution de Fermi-Dirac, résistance va dépendre de température \\
présents partout : téléphones, caméra, ordinateur...\\

\begin{remarques} \begin{itemize} 
\item donner des ordres de grandeur de gap pour différents matériaux
\item intérêt de la photodiode PIN : augmenter artificiellement la taille de la ZCE. Ainsi, la majorité des photons y est absorbée. De plus, cette région intrinsèque étant pure ($99.99 \%$ pour le silicium), la vitesse des porteurs y est significativement augmentée. En effet, ces derniers n’y subissent que très peu de collisions du fait de cette absence d’impureté
\item bandes de valence, conduction et interdite : en physique des solides, on résout équation de Schrödinger, potentiel périodique du cristal ; la fonction d'onde d'un électron dans le cristal
\item théorème de Bloch
\item énergie en fonction du vecteur d'onde k 
\item important à mettre : diagramme simplifique en fonction de k, paraboles
\item gap direct/gap indirect
\item pourquoi quand il y a des électrons dans bande de conduction il y a un courant électrique ? 
\item pourquoi parabolique ? car écart à électron libre $\frac{\hbar^2 k^2 }{2m^* }$
\item dissymétrie en fonction de k : 
\item énergie de Fermi d'un conducteur, peuplement des niveaux d'énergie
\item intrinsèque : non dopé ; utilisation : filtre optique (GaP), photorésistance (CdS), thermorésistance, sonde à effet Hall
\item $E_g = \frac{h c}{\lambda_g} = \frac{ 1.24 \left[ eV \right] }{ \lambda \left[ \mu m \right] }$
\item extrinsèque : dopé ; utilisation : transistor 
\item caractéristique de la photodiode : exponentielle ; zone génératrice, photovoltaïque : photopile ; si balayée trop rapidement (?), on observe des effets capacitifs
\item $V_0=0,7V$ pour le silicium
\item faire un courant alternatif : deux diodes en parallèle sens opposés
\item ordre de grandeur des températures de Fermi des métaux, conducteurs : $10^4 K$
\end{itemize} \end{remarques}