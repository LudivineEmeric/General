\Chapter{Lois de l'optique géométrique}

\nivpre{CPGE}{
 \begin{itemize}
  \item A
 \end{itemize}
}

\lemessage{Principes fondateurs de l'optique géométrique : pierres de base pour toute la conception optique, microscopie, télescopes et lunettes... Négliger interférences et diffraction : longueur à partir du cm. On en dégage des lois simples.}

\biblio{}

\section*{Introduction}
\section{Réflexion et réfraction}
\subsection{Lois de Snell-Descartes}
\subsection{Principe de Fermat}
\section{Systèmes optiques}
\subsection{Conditions de Gauss}
\subsection{Lentilles}
\section{Applications}
\subsection{Prisme}
\subsection{Mirages}
\subsection{Arc-en-ciel}
\section*{Conclusion}
