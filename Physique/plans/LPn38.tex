\Chapter{Equation de Schrödinger et applications}

\nivpre{CPGE}{
 \begin{itemize}
  \item optique ondulatoire
  \item 
  \item 
  \item 
 \end{itemize}
}

\lemessage{.}

\biblio{}

\section*{Introduction}
on a vu relation onde - corpuscule, longueur d'onde de De Broglie \\
expérience introductive : diffraction des électrons \\
accélération des électrons par tension  \\
sur graphite, diffraction comme un réseau en optique \\
matière est une onde ! \\
comment décrire ces états ? \\

Slide dualité onde-corpuscule : relation de De Broglie $\lambda_{db}=\frac{h}{p}$ ; pour des électrons accélérés : $\lambda_{db}=\frac{rd}{l}$ \\

%\marginpar{réf. \cite{ToutenunPC_2016},\cite{hecht},\cite{huard_pola}}

\section{État et évolution d'une particule}
\subsection{Fonction d'onde}
$\psi (M,t)$, valeurs complexes, interprétation : amplitude de probabilité de trouver la particule à un instant donné \\
Dans un volume $dV$ centré sur $M$ à l'instant $t$, la probabilité est : $dP=\lvert \psi(M,t) \lvert ^2  dV$ \\
caractère probabiliste \\
propriété importante, loi de conservation : intégré sur tout l'espace, on obtient $1$ puisqu'on sait que cette particule existe \\
 Comment étudier son évolution dans le temps ?  \\ 
\subsection{Équation de Schrödinger}
1925 \\
on étudie ici à 1D \\
$\psi=\psi_0 e^{i\omega t}$ \\
$E=\hbar \omega=\frac{p^2}{2m}+V$ ; $c=\lambda_{db} \nu$ \\
on part de l'équation de propagation et on démontre Schrödinger : $E \psi=-\frac{\hbar^2}{2m} \frac{\partial^2  \psi}{\partial x^2} +V(x) \psi$ \\
forme générale avec opérateur (pas démontrée ici) : $i\hbar \frac{\partial \psi}{\partial t}=-\frac{\hbar^2 }{2 m} \frac{\partial^2  \psi}{\partial x^2} + V(x) \psi $ \\
connaissance de l'évolution de la fonction d'onde\\
équation linéaire : si on a deux solutions, toute combinaison linéaire est alors solution \\

\section{Solutions}
\subsection{Cas libre}
$V=0$, on cherche des solutions stationnaires $\psi=\phi(x) f(t)$ \\
on les appelle états stationnaires \\
démonstration à partir de normalisation dans l'espace, et $f(t)=e^{-i\omega t}$ \\
$\psi ''+\frac{2m\omega}{\hbar} \psi$ =0 \\
$K=\sqrt{\frac{2m\omega}{\hbar}}$ : pulsation spatiale \\

\subsection{Cas d'une barrière de potentiel}
3 zones \\
énergie supérieure ou inférieure à $V_0$... \\

\section{Applications}

\subsection{Effet tunnel}

courant de probabilité (simplifié pour CPGE): $\vec{J}=\frac{\hbar \vec{k}}{m} \lvert \psi \lvert ^2$ \\

slide : radioactivité $\alpha$, He franchit une barrière de potentiel par effet tunnel \\
microscope à effet tunnel : sonder la matière à l'échelle nanométrique et manipulation d'atomes \\
\href{https://youtu.be/oSCX78-8-q0}{vidéo réalisée image par image}



\section*{Conclusion}
ne prend pas en compte effets relativistes 

\begin{remarques} 
Améliorations : 
\begin{itemize} 
\item plutôt introduire l'équation de Schrodinger à partir de l'énergie cinétique et énergie potentielle 
\item faire une seule particule
\item qqch de complètement différent de classique
\item probabilité de présence ? pas exactement, c'est qqch qui nous permet de déterminer la proba
\item champ de l'espace et du temps défini en tout point
\item aspect dynamique pour évolution temporelle : on a longueur d'onde de de broglie, vecteur d'onde, le même que Fourier ; consiste juste à établir relation de dispersion
\item on a psi à un moment donné : on intègre, alors on l'a à n'importe quel instant
\item on s'intéresse aux modes propres car solutions
\item la mesure est un phénomène probabiliste
\item hypothèses : $\psi$ et $\frac{d \psi}{dt}$ sont continus
\item puits de potentiel est plus intéressant pour les aspects de normalisation
\end{itemize}

Questions :
\begin{itemize} 
\item évolution d'un système quantique, tout système quantique ?   non
\item particules élémentaires : photon, boson de Higgs, quark, électrons... 
\item ça marche pour le photon ? non, pas de charge conservée, exemple corps noir : on crée des photons, oscillateurs
\item c'est quoi le modèle standard ? 
\item y a t'il des théories quantiques qui ne sont pas de la mécanique quantique ?
\item théorie quantique des champs ? système à nombres de particules variables, qui s'ajustent
\item au sens strict, la mécanique quantique relativiste n'existe pas 
\item passage à la mécanique classique ? théorème d'Ehrenfest ou approximation WKB avec paquets d'onde
\end{itemize} \end{remarques}