\Chapter{Instruments d'optique (hors microscopes)}

\nivpre{L3}{
 \begin{itemize}
  \item optique géométrique
  \item diffraction
 \end{itemize}
}

\lemessage{Un instrument optique s'utilise dans certaines conditions pour lesquelles il a été conçu. Il y a toujours des limitations, des compromis.}

\biblio{}

\section*{Introduction}
Les instruments d'optique sont omniprésents dans la vie quotidienne, par exemple pour la plupart d'entre nous possédons des lunettes. Mais il y a d’autres applications en physique plus fondamentale comme les télescopes. Mais aussi pour sonder la matière mais qu'on étudiera pas ici, les microscopes.
\marginpar{\cite{Optique_houard}}

Autres biblio possibles \cite{Optique_hecht}, \cite{Optique_perez}


\section{Appareil photo à focale fixe}
\subsection{Rappels sur les grandissements}
\subsection{Tirage de l'objectif et mise au point}
Def ; mise au point : action qui consiste à rendre l'image nette
\subsection{Profondeur et distance hyperfocale}

\section{Lunette astronomique}
\subsection{Caractéristiques}
Lunette : système afocal : objet à l'infini renvoi une image à l'infini
\subsection{Cercle oculaire}
Def ; image de l'objectif par l'oculaire, endroit où tous les rayons issus de l'objectif par loculaire se regroupent
Contient toute l'information sur l'objet, c’est l’endroit où l'on doit placer son œil

\section{Limitations}
\subsection{Limite de résolution}
\subsection{Aberrations}
chromatique : verre=milieu dispersif, utilisation 
géométriques : coma, astigmatisme et courbure de champ, distorsion

%\begin{ecran}
%	Contenu affiché sur diapositives
%\end{ecran}
%

%
%\transition{Belle transition entre deux parties ou sous-parties.}
%
%
%
%\begin{experience}
%	Expérience pour illustrer
%\end{experience}
%
%\begin{attention}
%	Erreur faite souvent ou point sur lequel insister.
%\end{attention}

\section*{Conclusion}


\begin{remarques}
	notion du grain du récepteur : pixel.. \\
	résolution \\
	CCD ou CMOS \\
	photo couleur : 3 pixels RGB, pixels peuvent être superposés  \\
	appareil photo infos : N, tirage et temps d'exposition \\
	téléobjectif : plusieurs lentilles, donc plusieurs diaphragmes, donc perte de luminosité et plus d'aberrations \\
	corriger aberration chromatique : doublets achromatique flint-crown \\
	corrgier aberration géométriques : diaphragmes pour centrer la lumière sur l’axe ; aberrations sphériques on peut aussi utiliser un doublet ; aberrations de coma il faut que les lentilles soient parallèles à l'axe optique. En pratique, toujours un compromis, optimisation, simulations Monte-Carlo \\
	télescope c'est fait avec des miroirs et une lunette avec des lentilles \\
	pas d'ab chromatique avec miroir \\
	télescope : meilleure résolution ; taille 20m pour le plus grands \\
	placés hors des villes pour éviter la pollution lumineuses et au sommet des montagnes pour diminuer les perturbations atmosphériques ; puis optique adaptative \\
\end{remarques}
