\Chapter{Symétries}

\nivpre{Licence}{
 \begin{itemize}
  \item électrostatique
  \item mécanique lagrangienne
  \item quadri-forces
  \item théorie de la bifurcation (pour diagrammes d'Elastica)
 \end{itemize}
}

\lemessage{Ici y a trop de choses, effet catalogue, ça part trop loin.}

\biblio{}

\section*{Introduction}
de tout temps physiciens se sont intéressés à symétrie \\
Euclide...

\section{Principe de symétrie}
\subsection{Qu'est-ce qu'on entend par symétrie}
symétrie planaire \\
Chimie : molécules chirales, prendre les formes à assembler \\
d'autres types de symétrie : symétries continues, associées à notions d'invariance \\
invariance par translation dans l'espace : ex lacher balle de ping-pong donne même résultat à un endroit ou à un autre \\
invariance par translation dans le temps : même expérience dans le temps \\

invariance par rotation \\

\subsection{Principe de Curie}
(Pierre Curie) \\
les effets d'un phénomène possèdent au moins les symétries de sa cause \\
électrostatique : cause distribution de charges $\rho$, effet : champ électrique $\vec{E}$ \\
symétries de $\rho$ d'une sphère chargée : ne dep pas du temps, inv par rotation $\rightarrow$ même chose pour $\vec{E}$ \\
résolution avec Gauss... \\
Remarque principe de Curie en chimie : mélange racémique, pour produire un seul énantiomère il faut briser la symétrie, introduire autre chose ; importance par exemple avec thalidomine tératogène vs anti-nauséeux, corps humain est lui même chiral \\


\subsection{Brisure de symétrie}
Elastica : flambage, position d'équilibre penchée (bifurcation) \\
Généralisation : il faut prendre en compte l'ensemble des effets possibles \\
diagramme de bifurcation : diagramme $x_{eq}$ en fonction de m \\ 
limitations ici : l'elastica est un peu déformé, tend vers un côté préférentiellement \\

\section{Symétries et lois de conservation}


\subsection{Conservation de l'impulsion}
$F=-\frac{\ddroit V}{\ddroit x}$ F dérive du potentiel V \\
particule de masse m dans un référentiel galiléen \\
PFD \\
hypothèse : le système est invariant par translation dans l'espace  \\
$\frac{\ddroit}{\ddroit x} \rightarrow 0$ \\
F=0 \\
$\frac{\ddroit(m\dot{x}}{\ddroit t} =0$ \\
l'impulsion se conserve \\

\subsection{Théorème de Noether}
lagrangien $L(x,\dot{x},t)$ avec x(t) et $\dot{x} (t)$ \\
hypothèse : le système est invariant par variation continue d'une grandeur S : $\frac{\ddroit L}{\ddroit S}=0$ \\

\subsection{Conservation de l'énergie}
invariance par translation dans le temps : $s=t$ \\
Hamiltonien...\\
calcul montre que $E_m$ se conserve \\



\section{Autres invariances}
\subsection{Invariance par rapport au choix de coordonnées}
la nature se fiche de notre choix de coordonnées \\
mécanique classique : force est un vecteur, $m\vec{a}$ \\
relativité restreinte : quadri-vecteurs \\

\subsection{Invariance du choix des unités}
 théorème $\Pi$ \\
 analyse dimensionnelle \\

dilatation des systèmes \\


\section*{Conclusion}
on peut en déduire bcp d'informations sur les propriétés que doit avoir un système \\
contraintes sur modèles théoriques \\
théorie quantique des champs \\


\begin{remarques} \begin{itemize} 
\item invariances et symétries : invariance est conséquence de symétrie
\item Curie a découvert le principe par cristallographie : cristaux ont propriétés de symétrie discrète ; effets : interaction avec champ électromagnétique, diffraction aux rayons X ; voir BUP 689
\item inversion du temps : important dans beaucoup de domaines, réversibilité
\item invariance par rotation vs symétrie par rotation
\item la chimie ne se résume pas aux effets chiraux, autres phénomènes en chimie qui jouent sur les réactions : effets stériques, orbitaux
\item manipulation sur coin de table : important pour montrer que c'est concret, analogies ; genre pendule dépend du temps mais invariant dans le temps
\item thermodynamique : transition de phase ferromagnétisme paramagnétisme $\rightarrow$ brisure de symétrie, bifurcations fourche
\item lien avec entropie : augmente lors de la brisure
\item x : coordonnées généralisée, par forcément une longueur
\item s : paramètre qui décrit la symétrie, nom ?
\item Hamiltonien est énergie du système toujours ? quand dimension d'une énergie oui
\item contexte historique Einstein : équation de Maxwell étaient invariant de Lorentz, pas Galilée, en relativité l'invariant est l'intervalle 
\item théorème $\Pi$ : si j'ai une fonction de variables dimensionnées, je peux le réécrire come une relation adimensionnée
\item symétrie de jauge
\item invariant de l'électromagnétisme et Noether : conservation de la charge
\item Elastica : imperfect pitchfork
\item faire calcul d'entropie ?
\item principe de relativité affirme que les lois physiques s'expriment de manière identique dans tous les référentiels inertiels : les lois sont « invariantes par changement de référentiel inertiel »
\end{itemize} \end{remarques}

